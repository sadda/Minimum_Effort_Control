\documentclass{article}


\usepackage[utf8]{inputenc}
\usepackage[english]{babel}
\usepackage{graphics}
\usepackage{amsfonts}
\usepackage{amssymb}
\usepackage{amsmath}
\usepackage{amsthm}
\usepackage{graphicx}
\usepackage{pgf, tikz}
\usetikzlibrary{patterns, arrows, automata, shapes, fit}
\usepackage{pgfplotstable}
\usepackage{todonotes}
\usepackage{comment}
\usepackage{bm}

\usepackage{pgfplots}

\usepackage[section]{algorithm}
\usepackage{algorithmicx}
\usepackage{algpseudocode}
\renewcommand{\algorithmicrequire}{\textbf{Input:}}
\renewcommand{\algorithmicensure}{\textbf{Output:}}
\algnewcommand\Offline{\item[\textbf{Offline:}]}

\newcommand{\R}{{\mathbb R}}
\newcommand{\eps}{\varepsilon}
\newcommand{\mnmz}{\operatorname*{minimize}}
\newcommand{\mxmz}{\operatorname*{maximize}}
\newcommand{\st}{\operatorname{subject\ to}}
\newcommand{\argmin}{\operatorname*{argmin}}

\newcommand{\norm}[1]{\|#1\|}
\newcommand{\nrm}[1]{|#1|}
\newcommand{\loss}{\operatorname{loss}}
\newcommand{\Xcal}{\mathcal{X}}
\newcommand{\Ycal}{\mathcal{Y}}


\newtheorem{thm}{Theorem}[section]
\newtheorem{theorem}[thm]{Theorem}
\newtheorem{lemma}[thm]{Lemma}
\newtheorem{hypothesis}[thm]{Hypothesis}
\newtheorem{proposition}[thm]{Proposition}
\newtheorem{corollary}[thm]{Corollary}

\theoremstyle{definition}
\newtheorem{definition}[thm]{Definition}
\newtheorem{example}[thm]{Example}
\newtheorem{remark}[thm]{Remark}
\newtheorem{assumption}[thm]{Assumption}


\newcommand{\argmax}{\operatorname{argmax}}
\newcommand{\rank}{\operatorname{rank}}


\usepackage{a4wide}



\makeatletter
\DeclareRobustCommand{\rvdots}{%
  \vbox{
    \baselineskip4\p@\lineskiplimit\z@
    \kern-\p@
    \hbox{.}\hbox{.}\hbox{.}
  }}
\makeatother


\newcommand{\relu}{\operatorname{ReLU}}


\begin{document}

Primal problem
\begin{equation}\label{eq:primal}
  \aligned
  \mnmz_x\qquad &\norm{x}_\infty \\
  \st\qquad &Ax=y.
  \endaligned
\end{equation}

Dual problem
\begin{equation}\label{eq:dual}
  \aligned
  \mxmz_u\qquad &y^\top u \\
  \st\qquad &\norm{A^\top u}_1 \le 1.
  \endaligned
\end{equation}

\begin{theorem}
  There is a finite set $U$ such that for any $y$, there exists some $u\in U$ such that $u$ is the solution of \eqref{eq:dual} for $y$. This $U$ equals to the set of extremal points of $\{u\mid \norm{A^\top u}_1\le 1\}$.
\end{theorem}


\begin{algorithm}
\begin{algorithmic}[1]
  \Offline Compute the set $U$ with finite number of elements
  \Require $y$ for which we need to solve \eqref{eq:primal}
  \State Select $\bar u \in \argmax_{u\in U}y^\top u$. It solves \eqref{eq:dual}
  \State Based on complementarity find the solution of \eqref{eq:primal}
\end{algorithmic}
\end{algorithm}


\appendix

\section{Appendix}

Computing directly the set of extremal points of the feasible set of \eqref{eq:dual} is complicated. However, there is a connected set, for which the computation is possible. We describe it in the following statement.



\begin{algorithm}
\caption{For finding extremal points of the set $\{v\mid Bv=b,\ v\ge 0\}$}
\label{alg:extremal}
\begin{algorithmic}[1]
  \Require Matrix $B$ of size $(m,n)$ with $\rank B=m$
  \State $V\gets\emptyset$
  \State Denote by $\mathcal I$ the set of all tuples of length $m$ selected from $1,\dots,n$ (without replacement).
  \For{$I\in\mathcal I$}
    \State Let $B_{\rm sub} := B_{\cdot,I}$ be the $(m,m)$ submatrix of $B$ with columns indexed by $I$
    \If{$\rank B_{\rm sub} = m$}
      \State Compute the unique solution $z_{\rm sub}$ of $B_{\rm sub}z_{\rm sub} = b$
      \If{$z_{\rm sub}\ge 0$}
        \State Define the vector $z$ with $0$ everywhere and $z_{\rm sub}$ on $I$
        \State $V\gets V\cup\{z\}$
      \EndIf              
    \EndIf
  \EndFor
  \Ensure $V$ as the set of extremal points of \eqref{eq:extremal_set}
\end{algorithmic}
\end{algorithm}





\begin{proposition}\label{prop:extremal}
  Consider matrix $B$ of size $(m,n)$ with $\rank B=m$ and any vector $b$. Then Algorithm \ref{alg:extremal} computes the set of all extremal points of
  \begin{equation}\label{eq:extremal_set}
    \{v\mid Bv=b,\ v\ge 0\}.
  \end{equation}
\end{proposition}




\begin{lemma}\label{lemma:dual_enh}
Dual problem is equivalent to
\begin{equation}\label{eq:dual_enh}
  \aligned
  \mxmz_{u^+,u^-,z^+,z^-,w}\qquad &y^\top u^+ - y^\top u^-  \\
  \st\qquad &A^\top u^+ - A^\top u^- - z^+ + z^- = 0, \\
  &\sum(z_i^+ + z_i^-) + w = 1, \\
  &u^+,u^-,z^+,z^-,w \ge0.
  \endaligned
\end{equation}
The solution $u$ of \eqref{eq:dual} can be recovered as $u = u^+ - u^-$.
\end{lemma}
\begin{proof}
  The constraint $\norm{A^\top u}_1 \le 1$ can be equivalently written by
  $$
    \sum_i \nrm{A^\top u}_i \le 1
  $$
  Now we use the standard trick and write the absolute value as the difference of its positive and negative part $z=z^+-z^-$ with its absolute value $\nrm{z}=z^++z^-$. This gives
  $$
    \aligned
    A^\top u &= z^+ - z^-, \\
    z^+, z^- &\ge 0, \\
    \sum_i (z_i^+ + z_i^-) &\le 1.
    \endaligned
  $$
  Since there is no non-negativity constraint on $u$, we use the same trick to write $u=u^+-u^-$ with $u^+,u^-\ge 0$. Finally, we change the inequality to inequality constraint by adding a slack variable $w\ge 0$. This gives \eqref{eq:dual_enh}.
\end{proof}

Now we can use Proposition \ref{prop:extremal} to find the extremal points of the feasible set of \eqref{eq:dual_enh}. Due obtain the form necessary for the proposition, we set
\begin{equation}\label{eq:enh_matrices}
  \aligned
  v &= \begin{pmatrix} u^+ & u^- & z^+& z^- & w \end{pmatrix}^\top, \\
  B &= \begin{pmatrix} A^\top & -A^\top & -I & I & 0 \\ 0 & 0 & 1^\top & 1^\top & 1 \end{pmatrix}, \\
  b &= \begin{pmatrix}0\\1 \end{pmatrix}.
  \endaligned
\end{equation}

Using Proposition \ref{prop:extremal} we can compute the extramal points of the feasible set of \eqref{eq:dual_enh}. This is done in the first two steps of Algorithm \ref{alg:extremal2}. These points have components $\begin{pmatrix} u^+ & u^- & z^+& z^- & w \end{pmatrix}$ while the feasible set of \eqref{eq:dual} is only in the $u$-space. Proposition \ref{prop:extremal} also says that to get to the $u$-space, we need to set $u=u^+-u^-$. This is done in the next two steps of Algorithm \ref{alg:extremal2}. However, this procedure is likely to create a superset of the set of extremal points of \eqref{eq:dual}. For this reason, we remove the points which lie in the interior (and thus, they cannot be extremal).

The next lemma proves that this procedure indeed finds all extremal points of \eqref{eq:dual}.

\begin{lemma}
  If $A$ has full row rank, then the set $U$ found by Algorithm \ref{alg:extremal2} is the set of extremal points of \eqref{eq:dual}.
\end{lemma}
\begin{proof}
  %Lemma \ref{lemma:dual_enh} says that the solution $u$ of \eqref{eq:dual} can be recovered as $u = u^+ - u^-$.
\ \\
\todo[inline]{Difficult.}
\end{proof}





\begin{algorithm}
\caption{For finding $U$}
\label{alg:extremal2}
\begin{algorithmic}[1]
  \Require Matrix $A$ of size $(k,l)$ with $\rank A=k$
  \State Set $B$ and $b$ as in \eqref{eq:enh_matrices}
  \State Use Algorithm \ref{alg:extremal} to compute $V$
  \State Enumerate $V = \{\begin{pmatrix} u^{+,i} & u^{-,i} & z^{+,i}& z^{-,i} & w^i \end{pmatrix}\}_{i=1}^I$
  \State Set $U = \{u^{+,i} - u^{-,i}\}_{i=1}^I$
  \State Remove all points from $U$ which are in the interior of its convex hull
  \Ensure $U$ as the set of extremal points of \eqref{eq:dual}
\end{algorithmic}
\end{algorithm}






\end{document}
