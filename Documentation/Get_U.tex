\documentclass{article}


\usepackage[utf8]{inputenc}
\usepackage[english]{babel}
\usepackage{graphics}
\usepackage{amsfonts}
\usepackage{amssymb}
\usepackage{amsmath}
\usepackage{amsthm}
\usepackage{graphicx}
\usepackage{pgf, tikz}
\usetikzlibrary{patterns, arrows, automata, shapes, fit}
\usepackage{pgfplotstable}
\usepackage{todonotes}
\usepackage{comment}
\usepackage{bm}

\usepackage{pgfplots}

\usepackage[section]{algorithm}
\usepackage{algorithmicx}
\usepackage{algpseudocode}
\renewcommand{\algorithmicrequire}{\textbf{Input:}}
\renewcommand{\algorithmicensure}{\textbf{Output:}}

\newcommand{\R}{{\mathbb R}}
\newcommand{\eps}{\varepsilon}
\newcommand{\mnmz}{\operatorname*{minimize}}
\newcommand{\mxmz}{\operatorname*{maximize}}
\newcommand{\st}{\operatorname{subject\ to}}
\newcommand{\argmin}{\operatorname*{argmin}}

\newcommand{\norm}[1]{\|#1\|}
\newcommand{\nrm}[1]{|#1|}
\newcommand{\loss}{\operatorname{loss}}
\newcommand{\Xcal}{\mathcal{X}}
\newcommand{\Ycal}{\mathcal{Y}}


\newtheorem{thm}{Theorem}[section]
\newtheorem{theorem}[thm]{Theorem}
\newtheorem{lemma}[thm]{Lemma}
\newtheorem{hypothesis}[thm]{Hypothesis}
\newtheorem{proposition}[thm]{Proposition}
\newtheorem{corollary}[thm]{Corollary}

\theoremstyle{definition}
\newtheorem{definition}[thm]{Definition}
\newtheorem{example}[thm]{Example}
\newtheorem{remark}[thm]{Remark}
\newtheorem{assumption}[thm]{Assumption}



\usepackage{a4wide}



\makeatletter
\DeclareRobustCommand{\rvdots}{%
  \vbox{
    \baselineskip4\p@\lineskiplimit\z@
    \kern-\p@
    \hbox{.}\hbox{.}\hbox{.}
  }}
\makeatother


\newcommand{\relu}{\operatorname{ReLU}}


\begin{document}

Primal problem
\begin{equation}\label{eq:primal}
  \aligned
  \mnmz_x\qquad &\norm{x}_\infty \\
  \st\qquad &Ax=y.
  \endaligned
\end{equation}

Dual problem
\begin{equation}\label{eq:dual}
  \aligned
  \mxmz_u\qquad &y^\top u \\
  \st\qquad &\norm{A^\top u}_1 \le 1.
  \endaligned
\end{equation}

Dual problem enhanced
\begin{equation}\label{eq:dual_enh}
  \aligned
  \mxmz_{u^+,u^-,z^+,z^-,w}\qquad &y^\top u^+ - y^\top u^-  \\
  \st\qquad &A^\top u^+ - A^\top u^- - z^+ + z^- = 0, \\
  &\sum(z_i^+ + z_i^-) + w = 1, \\
  &u^+,u^-,z^+,z^-,w \ge0.
  \endaligned
\end{equation}

Comments
\begin{itemize}\itemsep 0pt
 \item All three problems are equivalent and they are linear problems.
 \item Each linear problem has a solution in an extremal point (corner) of its feasible set. The number of extremal points is finite.
 \item The solution set of \eqref{eq:dual} does not depend on $y$. Denote the finite set of extremal points by $U$.
 \item Previous two bullets imply the following: For every $y$, there is always some $u\in U$ which solves \eqref{eq:dual}.
 \item I do not know how to compute the extremal points of \eqref{eq:dual} but there is a formula for computing the extremal points of \eqref{eq:dual_enh}.
\end{itemize}

This suggests that the way to go is to compute the extremal points of \eqref{eq:dual_enh}. Since they are in an enhanced space $(u^+,u^-,z^+,z^-,w)$, we reduce them into the original space $u$ corresponding to problem \eqref{eq:dual}. This reduction will create a superset of the extremal points of \eqref{eq:dual}. But there is a way of obtaining the set of extremal points of \eqref{eq:dual} from this superset.

\end{document}
